\documentclass[10pt, a4paper]{article}

% Text packages
\usepackage[french]{babel}
\usepackage[utf8x]{inputenc} % fix text encode to the final document
\usepackage{xspace} % fix spaces problems
% \usepackage{lmodern} % Latin founts, can be useful?

% Fonts packages
\usepackage{ifxetex}
\ifxetex
    \usepackage{fontspec}
    \setmainfont{OpenDyslexic}
    [
      BoldFont=OpenDyslexic-Bold,
      ItalicFont=OpenDyslexic-Italic,
      BoldItalicFont=OpenDyslexic-BoldItalic,
      SmallCapsFont=OpenDyslexicMono
    ]
    \setsansfont{OpenDyslexic}[
      Scale=MatchLowercase,
      BoldFont=OpenDyslexic-Bold,
      ItalicFont=OpenDyslexic-Italic,
      BoldItalicFont=OpenDyslexic-BoldItalic,
      SmallCapsFont=OpenDyslexicMono
    ]
    \setmonofont{OpenDyslexicMono}[
      Scale=MatchLowercase,
      BoldFont=OpenDyslexic-Bold,
      ItalicFont=OpenDyslexic-Italic,
      BoldItalicFont=OpenDyslexic-BoldItalic,
      SmallCapsFont=OpenDyslexicMono
    ]
\else
    \usepackage[T1]{fontenc}
\fi

% Math fonts
\usepackage{amsmath, amsfonts} % amssymb %INFO: amsmath < mathtools
\usepackage{latexsym}

% Theorem fonts
\usepackage{amsthm}

% Title import
\usepackage{authoraftertitle}

% Date time
\usepackage{datetime}

% Clickable link
\usepackage{hyperref}
\usepackage{bookmark}

% Image
\usepackage{graphicx}
\usepackage{subcaption} % Image side by side

% Import table
\usepackage{tabularx}
\usepackage{multirow} % merge rows
\usepackage{multicol} % merge columns
\usepackage[table]{xcolor}
\usepackage{longtable} % Allow tables to occupy several pages
\usepackage{array} % For table and array format

\usepackage{float} % For floating objects position lik figures or tables

% Text color
\usepackage{xcolor}

% Custom titles
% \usepackage{titlesec}

% Code import
% \usepackage{lstlisting}

% Graphviz import
% \usepackage[pdf]{graphviz}

% Page geometry (landscape...)
% \usepackage[driver=pdftex]{geometry}
% \usepackage{lscape}

% \usepackage{lipsum} % lorem ipsum

\usepackage{magictex}


\title{Rapport}
\author{GOEDEFROIT Charles}
\subject{Sur l'article : Arbitration Policies for On-Demand User-Level I/O Forwarding on HPC Platforms}
\keywords{}
\date{\today}

% Bibliography change type
\bibliographystyle{plain}

% Pdf and link setup
\hypersetup{
    colorlinks=true,
    citecolor=blue,
    filecolor=black,
    linkcolor=black,
    urlcolor=blue,
    pdftitle=\MyTitle,
    pdfauthor=\MyAuthor,
    pdfsubject=\MySubject,
    pdfkeywords=\MyKeywords,
    pdfpagemode=FullScreen
}

% Images / pdf folder
\graphicspath{ {./images/} }


\begin{document}

\begin{titlepage}
	\centering
  \ {} % important
	\vfill
	\vspace{1cm}
	{\scshape\Huge\MyTitle\par}
	\vspace{0.5cm}
	{\Large\MySubject\par}
	\vspace{1cm}
	\MyAuthor
	\vfill
	{\large\MyDate\par}
\end{titlepage}

\newpage


\section{L'objectif de l'article}
% TODO: the objective of the paper;

L'article s'intéresse à différante politique d'accés au noeux de données en fonction des
patternes d'accès au données par l'applicaiton.

L'article propose une politique basé sur le problème du sac à dos (Knapsack problem) à choix
multiple. Ce problem cherche à maximisé la bandpassat global en donnant plus de noeux I/O au
applications qui en on le plus besoin.

Il propose aussi une solution qui permet d'utilisé, à la demande, au niveaux utilisateur et
pandent l'execution, différantes politiques de \emph{I/O forwarding}.

Il montre que leur technique augment de façon transparent l'utilisation de la bande passante
globale jusque'à 85\% par rapport a la politique statique utilisé par défault. Il le montre
avec de nombreuse expérimentations et directement sur une infrastructure (live setup).



\section{Le context de l'article}
% - its scientific context or the background to understand the work;

Le \emph{I|O forwarding} est une technique très utilisé dans le HPC. Cette technique permet
d'augmenté les performance, en diminuent la contention, d'accès au serveurs de stockage de
données tous en étant transparent pour l'utilisateur.

évite les accé dirrect au machine de stockage paralléle

Le \emph{I|O forwarding} consiste à une groupe de noeux qui resoive les requétes de
l'applicaiton et les transmet au system de fichier paralléle. Permet aussi de changer la form
des requétes (Ex : au lieu de plusieurs petite requétes on en fait une seul grand).

La technique habituelle est d'assigné statiquement un noeux de stockage à un noeux de calcule.
Ces liens ce correspond pas tous le temps au transfére de données d'on l'appication à besoin.
Cette technique amméne à une mauvaise utilisation des resources.


\section{The problem}
% - a vision of the problem being treated (types of tasks and resources, objective function, constraints, etc), if possible;

\section{Explication des algo}
% - an explanation of the algorithms;


\section{Le perfs}
% - commentary regarding the performance evaluation, if present;


\section{Les autre publications}
% - information about other papers that reference it.


\section{Definition *}

La \emph{contention} : forte tentions sur un serveur (beaucoup de demandes).

\end{document}

% TODO: up to 8 pages
